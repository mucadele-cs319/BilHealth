\documentclass[a4paper, 12pt, titlepage]{article}

% Document quality things
\usepackage[utf8]{inputenc}
\usepackage{microtype, xcolor}
\usepackage{csquotes}
\usepackage{url, hyperref}
\hypersetup{colorlinks=true, linkcolor=black, citecolor=black, urlcolor=blue}

% Image-related packages
\usepackage{graphicx}
%\usepackage{float}
\graphicspath{{./gfx/}}
%\usepackage[font=small,skip=5pt]{caption}

% Setting margins
\usepackage[a4paper,bottom=2cm,top=2cm,left=2.5cm,right=2.5cm, includefoot]{geometry}

% Table helper packages
%\usepackage{multirow, multicol}
%\usepackage{makecell}
%\usepackage{array}
%\usepackage{tabularx} % Not needed currently, but has a few nice options
%\usepackage{wrapfig} % Floating figures/tables
\usepackage{booktabs}
\usepackage{longtable}

% Prevents spamming tedious newlines everywhere, also disables auto indentation, etc.
\usepackage[skip=0.75\baselineskip plus 2pt]{parskip}

% Self-explanatory
%\usepackage{titlesec}
%\titleformat{\section}[block]{\normalfont\scshape\Large}{\thesection}{1em}{}
%\titleformat{\subsection}{\normalfont\large}{\thesubsection}{1em}{}

\title{
    {CS 319 - Object Oriented Software Engineering}\\
    {\small Instructor: Eray Tüzün, TA: Muhammad Umair Ahmed}\\ % TODO: Add the other TA's name here
    {\vspace{10mm}BilHealth}\\
    {\Large \textbf{Design Report}}\\
    {\small Iteration 1}\\
    {\vspace{10mm}\includegraphics[width=0.3\linewidth]{bilkentlogo}}
}
\author{
  Mehmet Alper Çetin\\ \texttt{21902324}
  \and
  Vedat Eren Arıcan\\ \texttt{22002643}
  \and
  Uygar Onat Erol\\ \texttt{21901908}
  \and
  Recep Uysal\\ \texttt{21803637}
  \and
  Efe Erkan\\ \texttt{21902248}
}
\date{\today}

% \usepackage{tikz}
% \usetikzlibrary{automata, positioning, arrows}

\usepackage{pdflscape, pdfpages}

\begin{document}
  \maketitle
  \tableofcontents
  \pagebreak

  \section{Introduction}

  \subsection{Purpose of the System}

  \subsection{Design Goals}

  \section{High-Level Software Architecture}

  \subsection{Subsystem Decomposition}

  \subsection{Hardware-Software Mapping}

  The project does not have any extraordinary hardware requirements.
  As it is a web application, it requires the users to have a device using which they can
  connect to the web server of the application over the internet.
  The main target devices for a supported user experience are desktop computers, laptops, and mobile phones.

  In order to run the project the intended way, the host system's hardware must support stable containerization through Docker.
  It is possible to reconfigure certain entrypoints of the project to support bare-metal execution too, at the cost of
  system replicability and practicality.
  Similar to most other web services, the project is officially supported to run on a Linux environment, which
  may be of consideration when choosing hardware.

  \subsection{Persistent Data Management}

  The project makes use of PostgreSQL as its persistent data storage solution.
  The project does not have specific requirements that directly justify the use of PostgreSQL,
  however, this database system is well rounded enough to save the development process from any potential drawbacks.
  The \texttt{psql} command line tool is easy to use and supports our workflow in implementing database migrations.
  Also, the Docker images for the database are robust and simple,
  which has allowed the project to be supported by a containerized database.

  On top of the actual database, the project uses EF Core as an object relational mapper (ORM),
  which is Microsoft's official library for data persistence.
  We use the recommended \textit{code-first} approach of database management, which means our PostgreSQL tables are
  constructed by EF Core with respect to our entity definitions in C\# code.
  The EF Core toolchain generates migrations, which it also converts to SQL scripts,
  using which a database can be configured to host the project in moments.

  \subsection{Access Control and Security}

  \subsection{Boundary Conditions}

  \section{Low-Level Design}

  \subsection{Object Design Trade-Offs}

  \subsection{Final Object Design}

  \subsection{Packages}

  \begin{itemize}
    \item \texttt{Microsoft.AspNetCore}\textbf{:}\\
      This package is the framework supporting the entire web application.
    \item \texttt{Microsoft.EntityFrameworkCore.Design}\textbf{:}\\
      This package allows the EF Core migration tool to work on the project.
    \item \texttt{Microsoft.AspNetCore.Diagnostics.EntityFrameworkCore}\textbf{:}\\
      This package allows debugging EF Core migrations.
    \item \texttt{npgsql.EntityFrameworkCore.PostgreSQL}\textbf{:}\\
      This package is the database driver providing EF Core its interoperation with PostgreSQL.
    \item \texttt{Microsoft.AspNetCore.Identity.EntityFrameworkCore}\textbf{:}\\
      This package provides authentication and authorization of users through integration with EF Core.
    \item \texttt{BCrypt.Net-Next}\textbf{:}\\
      This package provides the BCrypt algorithms which are currently among the best in securing user passwords.
    \item \texttt{Microsoft.AspNetCore.Mvc.NewtonsoftJson}\textbf{:}\\
      This package is the most popular JSON library for .NET, used as the default solution in most cases.
    \item \texttt{Microsoft.AspNetCore.SpaProxy}\textbf{:}\\
      This package allows the .NET build process to launch and proxy with the client application.
  \end{itemize}

  \subsection{Class Diagrams of Layers}

  \subsubsection{Data Access Layer}

  \subsubsection{Business Logic Layer}

  \subsubsection{User Interface Layer}

  \section{Glossary}

\end{document}
