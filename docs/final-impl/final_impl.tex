\documentclass[a4paper, 12pt, titlepage]{article}

% Document quality things
\usepackage[utf8]{inputenc}
\usepackage{microtype, xcolor}
\usepackage{csquotes}
\usepackage{url, hyperref}
\hypersetup{colorlinks=true, linkcolor=black, citecolor=black, urlcolor=blue}

% Image-related packages
\usepackage{graphicx}
%\usepackage{float}
\graphicspath{{./gfx/}}
%\usepackage[font=small,skip=5pt]{caption}

% Setting margins
\usepackage[a4paper,bottom=2cm,top=2cm,left=2.5cm,right=2.5cm, includefoot]{geometry}

% Table helper packages
%\usepackage{multirow, multicol}
%\usepackage{makecell}
%\usepackage{array}
%\usepackage{tabularx} % Not needed currently, but has a few nice options
%\usepackage{wrapfig} % Floating figures/tables
\usepackage{booktabs}
\usepackage{longtable}

% Prevents spamming tedious newlines everywhere, also disables auto indentation, etc.
\usepackage[skip=0.75\baselineskip plus 2pt]{parskip}

% Self-explanatory
%\usepackage{titlesec}
%\titleformat{\section}[block]{\normalfont\scshape\Large}{\thesection}{1em}{}
%\titleformat{\subsection}{\normalfont\large}{\thesubsection}{1em}{}

\title{
    {CS 319 - Object Oriented Software Engineering}\\
    {\small Instructor: Eray Tüzün, TA: Muhammad Umair Ahmed \& Elgun Jabrayilzade}\\
    {\vspace{10mm}BilHealth}\\
    {\Large \textbf{Final Report}}\\
    {\small Implementation}\\
    {\vspace{10mm}\includegraphics[width=0.3\linewidth]{bilkentlogo}}
}
\author{
  Mehmet Alper Çetin\\ \texttt{21902324}
  \and
  Vedat Eren Arıcan\\ \texttt{22002643}
  \and
  Uygar Onat Erol\\ \texttt{21901908}
  \and
  Recep Uysal\\ \texttt{21803637}
  \and
  Efe Erkan\\ \texttt{21902248}
}
\date{\today}

% \usepackage{tikz}
% \usetikzlibrary{automata, positioning, arrows}

\usepackage{pdflscape, pdfpages}

\begin{document}
  \maketitle
  \tableofcontents
  \pagebreak

  \section{Introduction}

  Our project is a health center management software system, built in the form of a web application.
  The main goal is to provide online, remote attention to patients to increase productivity for all parties involved.
  The primary method of interaction is through \textit{case}s, which contain all relevant information for a given medical situation.
  Patients use the system to open cases and request appointments, and the health center staff acts on these requests
  to provide medical services.
  These interactions also function as a variety of medical records for the health center.

  \subsection{Current Implementation State}
  
  
  
  \section{Lessons Learned}
  
  First, we learned difficulties to find a team. 
  Finding teammates with qualities enough was a hard job, although it was made easier with peerpanda.net. 
  We do believe we have now more experience to finding requirements for a given task. 
  We see that going out there and dealing with real-life problems were different than given academic tasks. 
  We do we believe the hardship of getting requirements were environment problems, 
  such as the barrier between engineers and current users. 
  Not always users help and not always engineers understand their help.

  In terms of code, we learned how difficult it might to be entering a totally different style of coding. 
  Most of us didn’t used .NET up until now which made it harder to work. 
  Most of us didn’t know C# which made it hard for all of us. 
  We do believe some of us learned how to approach those new fields and some of us learned not to underestimate learning new fields and should allocate enough time.
  Design patterns were another important topic to be considered, 
  We do believe some of us learned the difficulties to execute rules of design patterns to an actual project code.

  We also had a little experience about how does a group project plays out. 
  Communicating with people has its difficulties and we do believe a decentralized group can be a problem to decide and do actions. 
  Since all of us had almost same background it might sometimes be hard to convince one of us sometimes when we think what we are doing is right.
  
  Lastly given a small amount of time we find out that not everything works out as we planned it be. 
  Problems accord and it was best to do something that fits into environment and time rather than what we think is the best solution.
  
  \pagebreak
  \section{Build \& Execution Instructions}
  
  There are two build flavors in which the application can run: development and production.
  The production build is more stable, but also requires more initial effort to run.
  Both builds are highly recommended to be run through \textbf{Docker}.
  
  The project runs best on a Linux-based environment, despite using Docker. 
  Through \href{https://docs.microsoft.com/en-us/windows/wsl/about}{WSL} and its integration with Docker,
  Windows is also able to run the project reasonably well.
  Note that the project source must be located within the WSL filesystem.
  In some instances, Docker causes WSL to consume unnecessarily high amounts of RAM.
  This can be solved through the WSL configuration
  \href{https://github.com/microsoft/WSL/issues/4166\#issuecomment-526725261}{as seen here}.
  
  For both of the build flavors, the prerequisites are that:
  \begin{itemize}
    \item you obtain the project source through tools such as \textit{git clone}.
    \item you have installed \href{https://docs.docker.com/get-docker/}{Docker}
      and \href{https://docs.docker.com/compose/install/}{docker-compose}.
  \end{itemize}
  
  Then, begin by changing your working directory to the root of the project directory.
  
  For the most up to date version of these instructions, see the README on the project's repository.
  
  \subsection{Development Build}
  
  \begin{enumerate}
    \item Build the Docker image:
      \begin{verbatim}
        docker-compose -f docker-compose.devel.yml build
      \end{verbatim}
    \item Any time after the first image build, run the application via:
      \begin{verbatim}
        docker-compose -f docker-compose.devel.yml up
      \end{verbatim}
  \end{enumerate}
  
  Once the containers are fully up, the project should be visible at \url{https://localhost:7257/}. 
  This development image mounts the project folder into the container, allowing changes to the code to be seen in real-time.
  
  Note that the HTTPS support is through self-signed certificates, and your web browser may warn you about it.
  Ideally, your browser should give you the option to add an exception for the certificate.
  This is completely safe to do.
  
  \pagebreak
  \subsection{Production Build}
  
  The production build process is slightly more involved, at least on the first run.
  You will need to have additionally installed
  \href{https://dotnet.microsoft.com/en-us/download/dotnet/6.0}{.NET Core SDK v6.0.x} and
  \href{https://docs.microsoft.com/en-us/ef/core/cli/dotnet\#installing-the-tools}{EF Core tools}.
  
  Then;
  
  \begin{verbatim}
# Build image
docker-compose -f docker-compose.prod.yml build

# Perform database migration (do this only on first run or new migration)
## Bring up only the database container
docker-compose -d -f docker-compose.prod.yml up dbpostgres
## Generate idempotent SQL script and copy into container
dotnet ef migrations --idempotent -o migrate.sql
docker cp migrate.sql bilhealth_db_postgres_1:/migrate.sql
## Enter the database environment and execute migration
docker exec -it bilhealth_dbpostgres_1 bash
cd / && psql -U postgres -d bilhealthprod -f migrate.sql && exit
## Bring down database
docker-compose -f docker-compose.prod.yml down

# Bring up all containers
docker-compose -d -f docker-compose.prod.yml up
  \end{verbatim}
  
  Once the containers are fully up, the project should be visible at \url{http://localhost:5000/}.
  Note that unlike the development image, this one may need to be rebuilt upon every change to the project.
  The database migration process needs to be done only if the database is not up to date with the latest migration.
  
  There is currently no support for HTTPS in the production build.
  If desired, a reverse proxy (such as Nginx) may be used to support HTTPS.
  
  \pagebreak
  \section{User's Guide}

  
  
  \pagebreak
  \section{Work Allocation}
  
  \begin{itemize}
    \item \textbf{Mehmet Alper Çetin:}
      Interview with the Bilkent University health center.
      Use Case, State and Activity Diagrams in Requirements Report.
      Lessons Learned in Final Report.
      Evaluation of Sibling Group's Design Report.
      Requirements Engineering.
    \item \textbf{Vedat Eren Arıcan:}
      Interview with the Bilkent University health center.
      Implementation of the backend and frontend code bases.
      DevOps tasks such as integrating Docker and a GitHub CI workflow.
      Writing textual report content through \LaTeX{}. Access matrix and deployment diagram.
    \item \textbf{Uygar Onat Erol:}
      Sequence Diagrams in Requirement Reports.
      Boundary Conditions and User Interface Layer Class Diagram in Design Report.
    \item \textbf{Recep Uysal:}
      State and Activity Diagrams in Requirement Reports.
      Design Goals and Object Design Trade-offs in Design Report.
    \item \textbf{Efe Erkan:}
      Use Case Diagram in Requirements Report.
      Evaluation of Sibling Group's Requirements Report.
      Implementation of the backend code bases.
      Subsystem Decomposition and Class Diagrams in Design Report.
  \end{itemize}

\end{document}
